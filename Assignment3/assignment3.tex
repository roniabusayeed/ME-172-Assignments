\documentclass[12pt, letterpaper]{report}
\usepackage[utf8]{inputenc}
\usepackage{amsmath}
\usepackage{amssymb}

\title{ME-172\\Computer Programming Language Sessional\\Assignment No. 03}
\author{Abu Sayeed Roni, ID: 1710065\\Dept: ME, Section: B-2}
\date{\today}

\begin{document}
\maketitle

\section*{Problem 1}
{\large Write a program to evaluate the sine series using for loop.}

\subsection*{Solution:}
The sine series is given by the mathematical formula:
\begin{displaymath}
\sin (x) = x - \frac{x^3}{3!} + \frac{x^5}{5!} - \frac{x^7}{7!} + ... 
\end{displaymath}
The following program asks the user to enter the value of $x$ and then it evaluates the sine series by adding up first $10$ terms.
\begin{verbatim}
    #include <stdio.h>
    #include <math.h>


    // Calculates the factorial of a non-negative integer.
    long factorial(int n)
    {
        long result = 1;
        for(; n > 1; n--)
        {
            result *= n;
        }
        return result;
    }

    int main(void)
    {
        // Take user input for x
        float x;
        printf("Enter x: ");
        scanf("%f", &x);

        // Represents the sum of sine series up to n terms.
        double sum = 0;

        // Total number of terms from start to add up.
        const int n = 10;

        // Calculate
        for(int i = 0, j = 0; j < n; i+= 2, j++)
        {
            sum += pow(-1, j) * pow(x, i + 1) / factorial(i + 1);
        }

        // Show output
        printf("Sum of sine series up to 10th term = %lf\n", sum);
    }


\end{verbatim}
Saving the file with name \texttt{sine.c}. Then Compiling the source file with the following command:

\begin{verbatim}
$ clang -o sine sine.c -lm
\end{verbatim}
An executable binary file with name \texttt{sine} will be created if everything goes right.
\subsubsection*{Output}
Running the executable with the following command:
\begin{verbatim}
$ ./sine
Enter x: 2
Sum of sine series up to 10th term = 0.909297
\end{verbatim}

\section*{Problem 2}
{\large Write a program that determines the number of trailing zeros at the end of $x!$ ($x$ factorial),
where $x$ is an arbitrary number. For instance, 5! is 120, so it has one trailing zero.}

\subsection*{Solution:}
The recursive definition of $x!$ is the following.
\begin{displaymath}
x! = x (x - 1)!
\end{displaymath}
And the iterative definition is the following. We will be using this definition in our code.
\begin{displaymath}
x! = x(x - 1)(x - 2)(x - 3)...3.2.1
\end{displaymath}

\begin{verbatim}
    #include <stdio.h>
    #include <stdbool.h>


    // Calculates the factorial of a non-negative integer.
    long factorial(int n)
    {
        long result = 1;
        for(; n > 1; n--)
        {
            result *= n;
        }
        return result;
    }

    int main(void)
    {
        // Take user input for x
        int x;
        printf("Enter x: ");
        scanf("%i", &x);

        long x_fact = factorial(x);
        long n = x_fact;

        // Represents the number of trailing zeros
        int count = 0;

        // Calculate
        for (;;)    // A while(true) would've been more appropriate
        {
            if (n % 10 == 0)
            {
                count++;
                n /= 10;
            }
            else
            {
                break;
            }
        }

        printf("%i! = %li\n", x, x_fact);
        printf("Total number of trailing zeros at the end of %i! = %i\n",
         x, count);
    }

\end{verbatim}
Saving the file with name \texttt{trailingZeros.c}. Then compiling the source file with the following command:
\begin{verbatim}
$ clang -o trailingZeros trailingZeros.c
\end{verbatim}
An executable binary file with name \texttt{trailingZeros} will be created if everything goes right.

\subsubsection*{Output}
Running the executable using the following command:
\begin{verbatim}
$ ./trailingZeros
Enter x: 10
10! = 3628800
Total number of trailing zeros at the end of 10! = 2
\end{verbatim}


\section*{Problem 3}
{\large Draw a Pascal’s triangle using C
programming (for loop)}

\subsection*{Solution:}
The binomial theorem is given by the following formula:

\begin{displaymath}
(a + b)^n = \sum_{k = 0}^n \binom{n}{k} a^{n - k}b^k
\end{displaymath}
If we just take the coefficients of $a^{n - k}b^k$ from the above binomial expansion formula and center-align them we get an
interesting structure what we call a Pascal's Triangle. From combinatorics, the definition of $\binom{n}{k}$ is given by the following formula: 
\begin{displaymath}
\binom{n}{k} = \frac{n!}{k!(n - k)!}
\end{displaymath}

\begin{verbatim}
    #include <stdio.h>


    // Calculates the factorial of a non-negative integer.
    long factorial(int n)
    {
        long result = 1;
        for(; n > 1; n--)
        {
            result *= n;
        }
        return result;
    }

    // Prints the coefficients of nth row. (counting from 0th row)
    void row(int n)
    {
        for(int k = 0; k <= n; k++)
        {
            int coe = factorial(n) / (factorial(k) * factorial(n - k));
            printf("%i ", coe);
        }
        printf("\n");
    }

    int main(void)
    {
        // Take user input for power of binomial, n.
        int n;
        printf("Enter n: ");
        scanf("%i", &n);

        // Print rows.
        for (int i = 0; i <= n; i++)
        {
            // Print preceding spaces
            for(int j = 0; j < n - i; j++)
            {
                printf(" ");
            }

            // Print row with height i.
            row(i);
        }
    }

\end{verbatim}
Saving the file with name \texttt{pascalsTriangle.c}. Then Compiling the source file with the following command:

\begin{verbatim}
$ clang -o pascalsTriangle pascalsTriangle.c
\end{verbatim}
An executable binary file with name \texttt{pascalsTriangle} will be created if everything goes right.
\subsubsection*{Output}
Running the executable with the following command:
\begin{verbatim}
$ ./pascalsTriangle
Enter n: 4
    1 
   1 1 
  1 2 1 
 1 3 3 1 
1 4 6 4 1 
\end{verbatim}
When a binomial raised to a power of $n$ is expanded, we get a total of $n+1$ terms from the exansion. The first row in our
output represents the coefficients of expansion of $(a + b)^0$, which is $1$. And if we keep going up to (and including) $(a + b)^4$ we get the coefficients represented by the last row in our output. And there are $(4 + 1) = 5$ coefficients (from $(4 + 1) = 5$ terms) in the last row, just like we preached.
\section*{}
NOTE: All the programs are written in \emph{Linux} environment. The executable binary files don't have an extension like .exe, .dmg, or .app beacuse in Linux whether a program is executable or not is determined by the permissions on the file, not the extension. And LLVM's \emph{Clang} is used for the compilation of above source files.
\end{document}
