\documentclass[12pt, letterpaper]{report}
\usepackage[utf8]{inputenc}
\usepackage{amsmath}

\title{ME-172\\Computer Programming Language Sessional\\Assignment No. 1}
\author{Abu Sayeed Roni, ID: 1710065\\Dept: ME, Section: B-2}
\date{February 24, 2020}

\begin{document}
\maketitle

\section*{Problem 1}
{\large Write a Progam to find the Area of a Circle. [NOTE: radius should be scanned from the keyboard]}

\subsection*{Solution:}
Formula used: 
\begin{displaymath}
A = \pi r^2
\end{displaymath}
Value of $\pi$ used in the following code is rounded to four decimal places. For our purposes this should do just fine.
\begin{verbatim}
    /**
     * Calculates the area of a circle
     */

    #include <stdio.h>
    #include <math.h>


    int main(void)
    {
        const float pi = 3.1416;
        float radius, area;

        // Take user input for radius.
        printf("Enter radius: ");
        scanf("%f", &radius);

        // Calculate area.
        area = pi * pow(radius, 2);

        // Print output.
        printf("Area: %.2f\n", area);
    }
\end{verbatim}
Saving the file with name \texttt{circle.c}. Then Compiling the source file with the following command:

\begin{verbatim}
$ clang -o circle circle.c -lm
\end{verbatim}
An executable binary file with name \texttt{circle} will be created if everything goes right.
\subsubsection*{Output}
Running the program with the following command:
\begin{verbatim}
$ ./circle

Enter radius: 3.5
Area: 38.48
\end{verbatim}

\section*{Problem 2}
{\large Write a Progam to compute average of four user given numbers (numbers can be of integer or floating types)}

\subsection*{Solution:}
Formula used:
\begin{displaymath}
\bar{x} = \dfrac{\sum\limits_{i=1}\limits^n x_i}{n}
\end{displaymath}
$n$, in our case, just happens to be $4$, though it could've been anything (any resonable positive integer) and we could still make the following code work just by setting \texttt{count} to whatever $n$ would've been. We have this flexibility because of the better design decision taken instead of just \emph{hard-coding} four separate variables.
\begin{verbatim}
    /**
     * Computes average of four user given numbers.
     */ 

    #include <stdio.h>


    int main(void)
    {
        // Allocating memory to store 4 numbers.
        const int count = 4;
        float numbers[count];

        // Prompt user for input
        printf("Enter four numbers: \n");
        for (int i = 0; i < count; i++)
        {
            scanf("%f", numbers + i);
        }

        // Calculate average
        float sum = 0;
        for (int i = 0; i < count; i++)
        {
            sum += numbers[i];
        }
        float avg = sum / count;

        // Print output
        printf("Average of ");
        for (int i = 0; i < count; i++)
        {
            if (i == count -1)
            {
                printf(", and ");
            }
            else if (i)
            {
                printf(", ");
            }
            printf("%.2f ", numbers[i]);
        }
        printf(" is: %.2f\n", avg);
    }
\end{verbatim}
Saving the file with name \texttt{average.c}. Then compiling the source file with the following command:
\begin{verbatim}
$ clang -o average average.c
\end{verbatim}
An executable binary file with name \texttt{average} will be created if everything goes right.

\subsubsection*{Output}
Running the program with the following command:
\begin{verbatim}
$ ./average
Enter four numbers:
1 2 3 4
Average of 1.00 , 2.00 , 3.00 , and 4.00  is: 2.50
\end{verbatim}

\section*{}
NOTE: All the programs are written in \emph{Linux} environment. The executable binary files don't have an extension like .exe, .dmg, or .app beacuse in Linux whether a program is executable or not is determined by the permissions on the file, not the extension. And LLVM's \emph{Clang} is used for the compilation of above source files which just happens to be my favourite.
\end{document}
