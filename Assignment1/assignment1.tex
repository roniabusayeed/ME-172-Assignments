\documentclass[12pt, letterpaper]{report}
\usepackage[utf8]{inputenc}

\title{ME-172\\Computer Programming Language Sessional}
\author{Abu Sayeed Roni, ID: 1710065\\Dept: ME, Section: B-2}
\date{\today}

\begin{document}
\maketitle

\section*{Problem 1}
{\large Write a Progam to find the Area of a Circle. [NOTE: radius should be scanned from the keyboard]}

\subsection*{Solution:}

\begin{verbatim}
    /**
     * Calculates the area of a circle
     */

    #include <stdio.h>
    #include <math.h>


    int main(void)
    {
        const float pi = 3.1416;
        float radius, area;

        // Take user input for radius.
        printf("Enter radius: ");
        scanf("%f", &radius);

        // Calculate area.
        area = pi * pow(radius, 2);

        // Print output.
        printf("Area: %.2f\n", area);
    }
\end{verbatim}

\section*{Problem 2}
{\large Write a Progam to compute average of four user given numbers (numbers can be of integer or floating types)}

\subsection*{Solution:}
\begin{verbatim}
    /**
     * Computes average of four user given numbers.
     */ 

    #include <stdio.h>


    int main(void)
    {
        // Allocating memory to store 4 numbers.
        const int count = 4;
        float numbers[count];

        // Prompt user for input
        printf("Enter four numbers: \n");
        for (int i = 0; i < count; i++)
        {
            scanf("%f", numbers + i);
        }

        // Calculate average
        float sum = 0;
        for (int i = 0; i < count; i++)
        {
            sum += numbers[i];
        }
        float avg = sum / count;

        // Print output
        printf("Average of ");
        for (int i = 0; i < count; i++)
        {
            if (i == count -1)
            {
                printf(", and ");
            }
            else if (i)
            {
                printf(", ");
            }
            printf("%.2f ", numbers[i]);
        }
        printf(" is: %.2f\n", avg);
    }

\end{verbatim}
\end{document}
