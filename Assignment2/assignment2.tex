\documentclass[12pt, letterpaper]{report}
\usepackage[utf8]{inputenc}
\usepackage{amsmath}

\title{ME-172\\Computer Programming Language Sessional\\Assignment No. 2}
\author{Abu Sayeed Roni, ID: 1710065\\Dept: ME, Section: B-2}
\date{\today}

\begin{document}
\maketitle

\section*{Problem 1}
{\large Write a C program to find the smallest of 3 integers taken
as input using nested if-else statement.}

\subsection*{Solution:}
The code below could be written a little more succintly if we're allowed to use logical operators
and if-else-if chaining. Or if the problem would've asked to find the smallest of, say, 100 integers,
we would've definitely used a loop to iterate over the array of integers keeping track of the current smallest
integer. But as the problem states that we are to use \emph{nested if-else} statement, here is the code.
\begin{verbatim}
    /**
     * Finds the smallest of three integers.
     */


    #include <stdio.h>


    int main(void)
    {
        // Take user input for three integers.
        int a, b, c;
        printf("%s\n", "Enter three integers:");
        scanf("%i %i %i", &a, &b, &c);

        int smallest;

        // Compare three integers and find the smallest one.
        if (a < b)
        {
            if (a < c)
            {
                smallest = a;
            }
            else
            {
                smallest = c;
            }
        }
        else
        {
            if (b < c)
            {
                smallest = b;
            }
            else
            {
                smallest = c;
            }
        }
        
        // Show output.
        printf("Smallest integer: %i\n", smallest);
    }

\end{verbatim}
Saving the file with name \texttt{smallest.c}. Then Compiling the source file with the following command:

\begin{verbatim}
$ clang -o smallest smallest.c
\end{verbatim}
An executable binary file with name \texttt{smallest} will be created if everything goes right.
\subsubsection*{Output}
Running the executable with the following command:
\begin{verbatim}
$ ./smallest
Enter three integers:
6
3
9
Smallest integer: 3
\end{verbatim}

\section*{Problem 2}
{\large Write a C program to find that whether a year is leap or
not.}

\subsection*{Solution:}
A leap year is a year, occurring once every four years, which has 366 days including 29 February as an intercalary day.
But then there is a catch: it cannot be a multiple of 100 because of some special rules that apply every 100 years.
But if the year is a multiple of 400, it is a leap year.

\begin{verbatim}
    /**
     * Checks if a given year is a leap year or not.
     */


    #include <stdio.h>
    #include <stdbool.h>


    int main(void)
    {
        // Take input from user for a year.
        int year;
        printf("%s", "Enter a year: ");
        scanf("%i", &year);

        // Check if the year is a leap year or not.
        bool leapYear;
        if (year % 400 == 0 || (year % 100 != 0 && year % 4 == 0))  
        {
            leapYear = true;
        }
        else 
        {
            leapYear = false;
        }

        // Show output.
        if (leapYear)
        {
            printf("%i is a leap year.\n", year);
        }
        else
        {
            printf("%i is not a leap year.\n", year);
        }
    }

\end{verbatim}
Saving the file with name \texttt{leapyear.c}. Then compiling the source file with the following command:
\begin{verbatim}
$ clang -o leapyear leapyear.c
\end{verbatim}
An executable binary file with name \texttt{leapyear} will be created if everything goes right.

\subsubsection*{Output}
Running the executable in the following manner a few times:
\begin{verbatim}
$ ./leapyear
Enter a year: 2019
2019 is not a leap year.

$ ./leapyear
Enter a year: 2020
2020 is a leap year.

$ ./leapyear
Enter a year: 2100
2100 is not a leap year.

$ ./leapyear
Enter a year: 2000
2000 is a leap year.
\end{verbatim}

\section*{}
NOTE: All the programs are written in \emph{Linux} environment. The executable binary files don't have an extension like .exe, .dmg, or .app beacuse in Linux whether a program is executable or not is determined by the permissions on the file, not the extension. And LLVM's \emph{Clang} is used for the compilation of above source files.
\end{document}
