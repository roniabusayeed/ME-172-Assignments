\documentclass[12pt, letterpaper]{report}
\usepackage[utf8]{inputenc}
\usepackage{amsmath}
\usepackage{amssymb}

\title{ME-172\\Computer Programming Language Sessional\\Assignment No. 2}
\author{Abu Sayeed Roni, ID: 1710065\\Dept: ME, Section: B-2}
\date{March 2, 2020}

\begin{document}
\maketitle

\section*{Problem 1}
{\large Write a C program to find the smallest of 3 integers taken
as input using nested if-else statement.}

\subsection*{Solution:}
The code below could be written a little more succintly if we're allowed to use logical operators
and if-else-if chaining. Or if the problem would've asked to find the smallest of, say, 100 integers,
we would've definitely used a loop to iterate over the array of integers keeping track of the current smallest
integer. But as the problem states that we are to use \emph{nested if-else} statement, here is the code.
\begin{verbatim}
    /**
     * Finds the smallest of three integers.
     */


    #include <stdio.h>


    int main(void)
    {
        // Take user input for three integers.
        int a, b, c;
        printf("%s\n", "Enter three integers:");
        scanf("%i %i %i", &a, &b, &c);

        int smallest;

        // Compare three integers and find the smallest one.
        if (a < b)
        {
            if (a < c)
            {
                smallest = a;
            }
            else
            {
                smallest = c;
            }
        }
        else
        {
            if (b < c)
            {
                smallest = b;
            }
            else
            {
                smallest = c;
            }
        }
        
        // Show output.
        printf("Smallest integer: %i\n", smallest);
    }

\end{verbatim}
Saving the file with name \texttt{smallest.c}. Then Compiling the source file with the following command:

\begin{verbatim}
$ clang -o smallest smallest.c
\end{verbatim}
An executable binary file with name \texttt{smallest} will be created if everything goes right.
\subsubsection*{Output}
Running the executable with the following command:
\begin{verbatim}
$ ./smallest
Enter three integers:
6
3
9
Smallest integer: 3
\end{verbatim}

\section*{Problem 2}
{\large Write a C program to find the roots of a quadratic equation $ax^2 + bx + c = 0$, that will take coefficients $a$, $b$, and $c$ as input and find the roots as output. Use nested if-else statement.}

\subsection*{Solution:}
Roots of a quadratic equation $ax^2 + bx + c = 0$, where the constants $a, b, c\in\mathbb{R}$ and $a \ne 0$ is given by the
following formula:
\begin{displaymath}
\frac{-b\pm\sqrt{b^2-4ac}}{2a}
\end{displaymath}
However, the nature of the roots are going to depend on the discriminant $b^2-4ac$. Here are the possible cases.
\begin{itemize}
    \item{If $b^2-4ac > 0$ then the roots of the quadratic equation $ax^2 +bx+ c = 0$ are real and unequal.}
    \item{If $b^2-4ac = 0$ then the roots of the quadratic equation $ax^2 +bx+ c = 0$ are real and equal.}
    \item{If $b^2-4ac < 0$ then the roots of the quadratic equation $ax^2 +bx+ c = 0$ are complex conjugate.}
    \item{If $b^2-4ac > 0$ and is a perfect square then the roots of the quadratic equation $ax^2 +bx+ c = 0$ are rational and unequal}
\end{itemize}
Now that we know what the roots are going to be like before we ever solved the equation for them, we can compute the \emph{third case} from above a bit differently than the other three cases.

If $b^2 - 4ac < 0$, then $-(b^2 - 4ac) = 4ac - b^2$ is a \emph{positive} quantity. Which will help us to resolve the formula into real and imaginary
parts like the following.
\begin{align*}
x &= \frac{-b\pm\sqrt{b^2-4ac}}{2a}\\
\Rightarrow x &= \frac{-b}{2a} \pm \frac{\sqrt{4ac - b^2}}{2a}i\\
\end{align*}
All we need to do now is translate this resolved formula into C and we are good to go!

\begin{verbatim}
    /**
     * Finds the roots of a quadratic equation.
     */


    #include <stdio.h>
    #include <math.h>
    #include <stdbool.h>


    int main(void)
    {
        // Take user input for coefficients a, b, and c.
        double a, b, c;
        printf("%s", "a = ");
        scanf("%lf", &a);
        printf("%s", "b = ");
        scanf("%lf", &b);
        printf("%s", "c = ");
        scanf("%lf", &c);

        // Calculate
        double discriminant = pow(b, 2) - 4 * a * c;
        bool hasRealRoots = discriminant < 0 ? false : true;
        double root1, root2;
        double realPart, imginaryPart;
        if (hasRealRoots)
        {
            root1 = (-b + sqrt(discriminant)) / (2 * a);
            root2 = (-b - sqrt(discriminant)) / (2 * a);
        }
        else
        {
            realPart = -b / (2 * a);

            // Discriminant, discriminant is negative here 
            // as hasRealRoots boolean variable is evaluated
            // false in the if clause. So, negating discriminant
            // will return a non-negative argument for sqrt function.

            // What we are actually doing is pulling out the 
            // imaginary i off the expression so that we can just
            // append it later when we will be showing output.
            imginaryPart = sqrt(-discriminant) / (2 * a);
        }

        // Show output
        printf("Roots are: \n");
        if (hasRealRoots)
        {
            printf("%.2lf, %.2lf\n", root1, root2);
        }
        else
        {
            // As the roots are complex conjugate, we write the 
            // real and imaginary parts separately which we
            // calculated into two separate variables beforehand.
            printf("%.2lf+%.2lf%c, %.2lf-%.2lf%c\n",
             realPart, imginaryPart, 'i', realPart, imginaryPart, 'i' );
        }
    }


\end{verbatim}
Saving the file with name \texttt{quadratic.c}. Then compiling the source file with the following command:
\begin{verbatim}
$ clang -o quadratic quadratic.c -lm
\end{verbatim}
An executable binary file with name \texttt{quadratic} will be created if everything goes right.

\subsubsection*{Output}
Running the executable using the following command:
\begin{verbatim}
$ ./quadratic
a = 1
b = 2
c = 3
Roots are: 
-1.00+1.41i, -1.00-1.41i
\end{verbatim}
The output is as expected which is equal to $-1+\sqrt{2}i$ and $-1-\sqrt{2}i$ repectively, except our output is rounded to
two decimal places.

\section*{}
NOTE: All the programs are written in \emph{Linux} environment. The executable binary files don't have an extension like .exe, .dmg, or .app beacuse in Linux whether a program is executable or not is determined by the permissions on the file, not the extension. And LLVM's \emph{Clang} is used for the compilation of above source files.
\end{document}
